\documentclass{beamer}

\mode<presentation>
{
  \usetheme{Copenhagen}

  \setbeamercovered{transparent}

}

\beamertemplatenavigationsymbolsempty

\definecolor{f2elblue}{RGB}{97,97,97}%{49,72,87}
\definecolor{dark-green}{RGB}{40,150,21}
\definecolor{light-gray}{gray}{0.85}
\definecolor{med-gray}{gray}{0.75}

\setbeamercolor{title in head/foot}{bg=orange!90!black,fg=white}
\setbeamercolor{author in head/foot}{bg=black,fg=white}
\setbeamercolor{subsection in head/foot}{bg=black!95,fg=white}
\setbeamercolor{structure}{fg=black}

\setbeamerfont{frametitle}{family=\rmfamily,shape=\itshape,size={\fontsize{14}{12}}}

\setbeamertemplate{blocks}[rounded][shadow=false]
\setbeamertemplate{items}[circle]

\setbeamertemplate{footline}%
{%
  \leavevmode%
  \hbox{%
    \begin{beamercolorbox}[wd=.45\paperwidth,ht=2.9ex,dp=1.125ex,leftskip=.3cm plus1fill,rightskip=.3cm]{author in head/foot}%
      \usebeamerfont{author in head/foot}\insertdate%
    \end{beamercolorbox}%
    \begin{beamercolorbox}[wd=.51\paperwidth,ht=2.9ex,dp=1.125ex,leftskip=.3cm,rightskip=.3cm plus1fil]{title in head/foot}%
      \usebeamerfont{title in head/foot}\insertshorttitle%
    \end{beamercolorbox}}%
    \begin{beamercolorbox}[wd=.04\paperwidth,ht=2.9ex,dp=1.125ex,center,leftskip=3pt plus1fill,rightskip=4pt]{author in head/foot}%
      \usebeamerfont{author in head/foot}\insertframenumber%
    \end{beamercolorbox}%
  \vskip0pt%
}%


\usepackage[english]{babel}

\usepackage[latin1]{inputenc}

\usepackage{times}
\usepackage[T1]{fontenc}

\usepackage{ulem}

\author{Philip Tellis / \texttt{philip@bluesmoon.info}}

\title{Messing with JS \& the DOM to analyse the Network}

\date{\href{http://jsconf.eu/2011/}{JSConf.eu} / 2011-10-01}


\pgfdeclareimage[height=0.5cm]{bluesmoon-logo}{../ui/default/bodybg}
\pgfdeclareimage[height=11pt]{cc-licence}{../by-nc-sa-3.0-88x31}
\logo{\pgfuseimage{cc-licence}}


%%%% Begin defining new commands for this template

% Serial number in slide title
\newcommand{\sn}[1]{\textrm{\textit{\Huge{\raisebox{-3pt}[4pt][8pt]{\textcolor{f2elblue}{#1}}}}}\hspace{4pt}}
% large Centered Italic Roman text within a slide (building block for the following)
\newcommand{\innersplash}[1]{
  \begin{center}
    \large \textrm{\textit{ #1 } }
  \end{center}
}
% Slide that has just one large centered thing to say
\newcommand{\splashslide}[2][{}]{
  \begin{frame}
  \frametitle{#1}
  \innersplash{#2}
  \end{frame}
}
% Leadin slide at the start of an exploit
\newcommand{\leadinslide}[2]{
  \splashslide{
     {\fontsize{150}{20}\selectfont{\raisebox{0pt}[90pt][0pt]{\textcolor{light-gray}{#1}}}} \\ \huge \textcolor{gray}{#2}
  }
}
% Suggested fix at the end of an exploit
\newcommand{\suggestedfix}[2]{
  \splashslide[\sn{#1} Suggested fix]{{#2}}
}
% Syntax highlighting
\def\brown<#1>#2{\textcolor<#1>{brown}{\textbf<#1>{#2}}}
\def\green<#1>#2{\textcolor<#1>{dark-green}{\textbf<#1>{#2}}}
\def\blue<#1>#2{\textcolor<#1>{blue}{\textbf<#1>{#2}}}
\def\purple<#1>#2{\textcolor<#1>{purple}{\textbf<#1>{#2}}}
\def\red<#1>#2{\textcolor<#1>{red}{\textbf<#1>{#2}}}

%%%% End of new commands




\begin{document}

\begin{frame}
  \titlepage
\end{frame}


\setbeamertemplate{background}{
  \parbox[c][.9\paperheight]{\paperwidth}{
    \vfill \pgfimage[width=\paperwidth]{www-invention}
  }
}
\begin{frame}{HTTP/TCP/IP}
\vspace{-4cm}
Developed by three guys with one beard between them
\end{frame}

\setbeamertemplate{background}{
  \parbox[c][.9\paperheight]{\paperwidth}{
    \vfill \pgfimage[width=\paperwidth]{discovery-of-json}
  }
}
\begin{frame}{JavaScript: The good parts}
\vspace{-4cm}
Discovered by one guy with one beard on him
\end{frame}
\setbeamertemplate{background}{}

\setbeamertemplate{background}{
  \parbox[c][0.9\paperheight]{\paperwidth}{
    \vfill \hfill \pgfimage[height=\paperheight]{../bluesmoon}
  }
}
\begin{frame}
Let's play with the nasty parts
\end{frame}
\setbeamertemplate{background}{}

\leadinslide{1}{Latency}

\setbeamertemplate{background}{
  \parbox[c][.6\paperheight]{\paperwidth}{
    \vfill \pgfimage[width=\paperwidth]{AC1}
  }
}
\begin{frame}
\frametitle{\sn{1}Blinkenlichten}
\vspace{4cm}
\begin{center}
It takes about 23ms for light to get from Berlin to NY \\
(35ms through fibre)
\end{center}
\end{frame}

\setbeamertemplate{background}{
  \parbox[c][.905\paperheight]{.93\paperwidth}{
    \vfill \hfill \pgfimage[height=.762\paperheight]{http-get}
  }
}
\begin{frame}{\sn{1}HTTP}
\end{frame}
\setbeamertemplate{background}{}

\splashslide{So to measure latency, we need to send 1 packet each way, and time it}

\begin{frame}[fragile]
\frametitle{\sn{1} Network latency in JavaScript}
\begin{semiverbatim}
  \green<1>{var} ts, rtt, img = \green<1>{new} \blue<1>{Image};
  img.onload=\green<1>{function}() \{ rtt=(+\green<1>{new} \blue<1>{Date} - ts) \};
  ts = +\green<1>{new} \blue<1>{Date};
  img.src="/1x1.gif";
\end{semiverbatim}
\end{frame}

\leadinslide{2}{TCP handshake}

\setbeamertemplate{background}{
  \parbox[c][.9\paperheight]{.8\paperwidth}{
    \vfill \hfill \pgfimage[height=.8\paperheight]{tcp-handshake}
  }
}
\begin{frame}{\sn{2}ACK-ACK-ACK}
\end{frame}

\setbeamertemplate{background}{
  \parbox[c][.905\paperheight]{.93\paperwidth}{
    \vfill \hfill \pgfimage[height=.762\paperheight]{http-keepalive}
  }
}
\begin{frame}{\sn{2}Connection: keep-alive}
\end{frame}
\setbeamertemplate{background}{}

\begin{frame}[fragile]
\frametitle{\sn{2} Network latency \& TCP handshake in JavaScript}
\vspace{-.3cm}
\begin{semiverbatim}
\green<1>{var} t=[], n=2, tcp, rtt;
\green<1>{var} ld = \green<1>{function}() \{
   t.push(+\green<1>{new} \blue<1>{Date});
   \green<1>{if}(t.length > n) 
     done();
   \green<1>{else} \{
     \green<1>{var} img = \green<1>{new} \blue<1>{Image};
     img.onload = ld;
     img.src="/1x1.gif?" + \blue<1>{Math}.random()
                         + '=' + \green<1>{new} \blue<1>{Date};
   \}
\};
\green<1>{var} done = \green<1>{function}() \{
  rtt=t[2]-t[1];
  tcp=t[1]-t[0]-rtt;
\};
ld();
\end{semiverbatim}
\end{frame}

\splashslide{The astute reader will notice that we've ignored DNS lookup time here... how would you measure it?}

\begin{frame}{\sn{1}Notes}
\begin{itemize}
  \item 1x1 gif is 35 bytes
  \item including HTTP headers, is smaller than a TCP packet
  \item Fires onload on all browsers
  \item 0 byte image fires onerror
  \item which is indistinguishable from network error
\end{itemize}
\end{frame}

\leadinslide{3}{Network Throughput}

\splashslide[\sn{3}Measuring Network Throughput]{\Large{\( \frac{size}{download\_time} \)}}

\begin{frame}[fragile]
\frametitle{\sn{3} Network Throughput in JavaScript}
\begin{semiverbatim}
\brown<1>{// Assume global object
// image=\{ url: ..., size: ... \}}
\green<1>{var} ts, rtt, bw, img = \green<1>{new} \blue<1>{Image};
img.onload=\green<1>{function}() \{
   rtt=(+\green<1>{new} \blue<1>{Date} - ts);
   bw = image.size*1000/rtt;    \brown<1>{// rtt is in ms}
\};
ts = +\green<1>{new} \blue<1>{Date};
img.src=image.url;
\end{semiverbatim}
\end{frame}

\splashslide[\sn{3}Measuring Network Throughput]{If it were that simple, I wouldn't be doing a talk at @jsconfeu}

\setbeamertemplate{background}{
  \parbox[c][.905\paperheight]{.85\paperwidth}{
    \vfill \hfill \pgfimage[height=.762\paperheight]{tcp-windows}
  }
}
\begin{frame}{\sn{3}TCP Slow Start}
\end{frame}
\setbeamertemplate{background}{}

\splashslide[\sn{3}Measuring Network Throughput]{So to make the best use of bandwidth, we need resources that fit in a TCP window}

\setbeamertemplate{background}{
  \parbox[c][.8\paperheight]{.85\paperwidth}{
    \vfill \hfill \pgfimage[height=.6\paperheight]{bandwidth-latency}
  }
}
\begin{frame}{\sn{3}There is no single size that will tax all available networks }
\end{frame}
\setbeamertemplate{background}{}

\begin{frame}[fragile]
\frametitle{\sn{3} Network Throughput in JavaScript -- Take 2}
\begin{semiverbatim}
\brown<1>{// image object is now an array of multiple images}
\green<1>{var} i=0;
\green<1>{var} ld = \green<1>{function}() \{
   \green<1>{if}(i>0)
      image[i-1].end = +\green<1>{new} \blue<1>{Date};
   \green<1>{if}(i >= image.length)
      done();
   \green<1>{else} \{
      \green<1>{var} img = \green<1>{new} \blue<1>{Image};
      img.onload = ld;
      image[i].start = +\green<1>{new} \blue<1>{Date};
      img.src=image[i].url;
   \}
   i++;
\};
\end{semiverbatim}
\end{frame}

\splashslide[\sn{3}Measuring Network Throughput]{Slow network connection, meet really huge image}

\begin{frame}[fragile]
\frametitle{\sn{3} Network Throughput in JavaScript -- Take 3}
\begin{semiverbatim}
   \green<1>{var} img = \green<1>{new} \blue<1>{Image};
   img.onload = ld;
   image[i].start = +\green<1>{new} \blue<1>{Date};
   image[i].timer =
         setTimeout(\green<1>{function}() \{
                       image[i].expired=true
                    \},
                    image[i].timeout);
   img.src=image[i].url;
\end{semiverbatim}
\end{frame}

\begin{frame}[fragile]
\frametitle{\sn{3} Network Throughput in JavaScript -- Take 3}
\begin{semiverbatim}
\green<1>{if}(i>0) \{
   image[i-1].end = +\green<1>{new} \blue<1>{Date};
   clearTimeout(image[i-1].timer);
\}
\end{semiverbatim}
\end{frame}

\begin{frame}[fragile]
\frametitle{\sn{3} Network Throughput in JavaScript -- Take 3}
\begin{semiverbatim}
\green<1>{if}(i >= image.length
      || (i > 0 \&\& image[i-1].expired)) \{
   done();
\}
\end{semiverbatim}
\end{frame}

\splashslide[\sn{3}Measuring Network Throughput]{Are we done yet?\\ \only<2->{sure...}}

\splashslide[\sn{3}Measuring Network Throughput]{Except network throughput is different every time you test it}

% http://www.flickr.com/photos/sophistechate/4264466015/
\setbeamertemplate{background}{
  \parbox[c][\paperheight]{.8\paperwidth}{
    \hfill \pgfimage[height=\paperheight]{statistician}
  }
}
\begin{frame}{Statistics to the Rescue}
\vspace{7.8cm}
\hspace{-1cm} \tiny \href{http://www.flickr.com/photos/sophistechate/4264466015/}{flickr/sophistechate/4264466015/}
\end{frame}
\setbeamertemplate{background}{}

\splashslide[\sn{3}Measuring Network Throughput]{The code for that is NOT gonna fit on a slide}

\leadinslide{4}{DNS}

\splashslide[\sn{4}Measuring DNS]{ \( time\_with\_dns - time\_without\_dns \) }

\begin{frame}[fragile]
\frametitle{\sn{4} Measuring DNS in JavaScript}
\vspace{-.3cm}
\begin{semiverbatim}
\green<1>{var} t=[], dns, ip, hosts=['http://hostname.com/',
                          'http://ip.ad.dr.ess/'];
\green<1>{var} ld = \green<1>{function}() \{
   t.push(+\green<1>{new} \blue<1>{Date});
   \green<1>{if}(t.length > hosts.length) 
     done();
   \green<1>{else} \{
     \green<1>{var} img = \green<1>{new} \blue<1>{Image};
     img.onload = ld;
     img.src=hosts[t.length-1] + "/1x1.gif";
   \}
\};
\green<1>{var} done = \green<1>{function}() \{
  ip=t[2]-t[1];
  dns=t[1]-t[0]-ip;
\};
ld();
\end{semiverbatim}
\end{frame}

\splashslide[\sn{4}Measuring DNS]{ What if DNS were already cached? \\ \only<2->{Use a wildcard DNS entry} }

\splashslide[\sn{4}Measuring DNS]{ What if you map DNS based on geo location? \\ \only<2->{A little more complicated, but doable} }

\splashslide[\sn{4}Measuring DNS]{ Full code in boomerang's DNS plugin }

\leadinslide{5}{IPv6}

\begin{frame}{\sn{5}Measuring IPv6 support and latency}
\begin{enumerate}
  \item Try to load image from IPv6 host
  \begin{itemize}
    \item If timeout or error, then no IPv6 support
    \item If successful, then calculate latency and proceed
  \end{itemize}
  \item Try to load image from hostname that resolves only to IPv6 host
  \begin{itemize}
    \item If timeout or error, then DNS server doesn't support IPv6
    \item If successful, calculate latency
  \end{itemize}
\end{enumerate}
\end{frame}

\splashslide[\sn{5}Measuring IPv6 support and latency]{ Full code in boomerang's IPv6 plugin }

\leadinslide{6}{Private Network Scanning}

\splashslide[\sn{6}Private Network Scanning]{JavaScript in the browser runs with the User's Security Privileges}

\splashslide[\sn{6}Private Network Scanning]{This means you can see the user's private LAN}

\begin{frame}{\sn{6}Private Network Scanning -- Gateways}
  \begin{enumerate}
    \item Look at common gateway IPs: 192.168.1.1, 10.0.0.1, etc.
    \item When found, look for common image URLs assuming various routers/DSL modems
    \item When found, try to log in with default username/password \\ if you're lucky, the user is already logged in
  \end{enumerate}
\end{frame}

\begin{frame}{\sn{6}Private Network Scanning -- Services}
  \begin{enumerate}
    \item Scan host range on private network for common ports (80, 22, 3306, etc.)
    \item Measure how long it takes to \texttt{onerror}
    \begin{itemize}
      \item Short timeout: connection refused
      \item Long timeout: something listening, but it isn't HTTP
      \item Longer timeout: probably HTTP, but not an image
    \end{itemize}
    \item Try an iframe instead of an image
  \end{enumerate}
\end{frame}

\leadinslide{--}{.done()}

\begin{frame}{Code/References}
  \begin{itemize}
    \item \texttt{http://github.com/bluesmoon/boomerang}
    \item \texttt{http://samy.pl/mapxss/}
  \end{itemize}
\end{frame}

\setbeamertemplate{background}{
  \parbox[c][\paperheight]{\paperwidth}{
    \vfill \hfill \pgfimage[height=\paperheight]{../bluesmoon}
  }
}
\begin{frame}
  \begin{itemize}
  \item Philip Tellis
  \item \href{http://lognormal.com/}{LogNormal.com}
  \item \small{\texttt{philip@bluesmoon.info}}
  \item \href{http://twitter.com/bluesmoon}{@bluesmoon}
  \item geek paranoid speedfreak
  \item \href{http://bluesmoon.info/}{http://bluesmoon.info/}
  \end{itemize}
\end{frame}
\setbeamertemplate{background}{}

\splashslide{\Huge Thank you }

\end{document}


