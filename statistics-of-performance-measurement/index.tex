\documentclass{beamer}

\mode<presentation>
{
  \usetheme{Warsaw}

  \setbeamercovered{transparent}

}

\beamertemplatenavigationsymbolsempty

\setbeamercolor{title in head/foot}{bg=orange!90!black,fg=white}
\setbeamercolor{subsection in head/foot}{bg=black!95,fg=white}
\setbeamercolor{structure}{fg=black}

\setbeamertemplate{blocks}[rounded][shadow=false]

\setbeamertemplate{footline}%
{%
  \leavevmode%
  \hbox{\begin{beamercolorbox}[wd=.5\paperwidth,ht=2.5ex,dp=1.125ex,leftskip=.3cm plus1fill,rightskip=.3cm]{author in head/foot}%
    \usebeamerfont{author in head/foot}\insertdate%
  \end{beamercolorbox}%
  \begin{beamercolorbox}[wd=.5\paperwidth,ht=2.5ex,dp=1.125ex,leftskip=.3cm,rightskip=.3cm plus1fil]{title in head/foot}%
    \usebeamerfont{title in head/foot}\insertshorttitle%
  \end{beamercolorbox}}%
  \vskip0pt%
}%

\usepackage[english]{babel}

\usepackage[latin1]{inputenc}

\usepackage{times}
\usepackage[T1]{fontenc}

\usepackage{ulem}

\author{Philip Tellis / \texttt{philip@bluesmoon.info}}

\title{The Statistics of Web Performance}

\date{\href{http://confoo.ca/}{ConFoo} / 2010-03-12}


\pgfdeclareimage[height=0.5cm]{bluesmoon-logo}{../ui/default/bodybg}
\pgfdeclareimage[height=11pt]{cc-licence}{../by-nc-sa-3.0-88x31}
\logo{\pgfuseimage{cc-licence}}





\begin{document}

\begin{frame}
  \titlepage
\end{frame}


\setbeamertemplate{background}{
  \parbox[c][\paperheight]{\paperwidth}{
    \hfill \pgfimage[height=\paperheight]{../bluesmoon}
  }
}
\begin{frame}{\textit{\$ finger philip}}
  \begin{itemize}
  \item Philip Tellis
  \item \small{\texttt{philip@bluesmoon.info}}
  \item @bluesmoon
  \item yahoo
  \item geek
  \end{itemize}
\end{frame}
\setbeamertemplate{background}{}


\section{Introduction}
\begin{frame}{}
  \begin{center} \LARGE{Introduction} \end{center}
\end{frame}

\subsection{The goal}
\begin{frame}{}
  \begin{itemize}
  \item Accurately measure page performance
    \begin{itemize}
    \item<2-> At least, as accurately as possible
    \end{itemize}
  \end{itemize}
\end{frame}

\begin{frame}{Be unintrusive}
If you try to measure something accurately, you will change something related

\hfill \tiny{-- Heisenberg's uncertainty principle}
\end{frame}

\begin{frame}{}
  \begin{center}
  And one number to rule them all
  \end{center}
\end{frame}

\subsection{Performance Measurement}
\begin{frame}{Bandwidth}
  \begin{itemize}
  \item Real bandwidth v/s advertised bandwidth
  \item Bandwidth to your server, not to the ISP
  \item Bandwidth during \textit{normal} internet usage
    \begin{itemize}
    \item<2-> If the user's always watching movies, you're not winning
    \end{itemize}
  \end{itemize}
\end{frame}

\begin{frame}{Latency}
  \begin{itemize}
  \item How long does it take a byte to get to the user?
    \begin{itemize}
    \item<2-> Wired, wireless, mobile, satellite?
    \item<2-> How many hops in between?
    \item<2-> Speed of light is constant
    \end{itemize}
  \item<3-> This is not a battle we will soon win.
    \begin{itemize}
    \item When was the last time you heard \textit{latency} mentioned in a TV ad?
    \end{itemize}
  \end{itemize}
  \href{http://www.stuartcheshire.org/rants/Latency.html}{\scriptsize{http://www.stuartcheshire.org/rants/Latency.html}}
\end{frame}

\begin{frame}{User perceived page load time}
  \begin{itemize}
  \item Time from ``click on a link'' to ``spinner stops spinning''
  \item This is what users notice
    \begin{itemize}
    \item Depends on how long your page takes to build
    \item Depends on what's in your page
    \item Depends on how long components take to load
    \item Depends on how long the browser takes to execute and render
    \end{itemize}
  \end{itemize}
\end{frame}

\begin{frame}{}
\begin{center}
We need to measure real user data
\end{center}
\end{frame}

\begin{frame}{}
\begin{center}
The statistics apply to any kind of performance data though
\end{center}
\end{frame}

\section{Statistics - I}

\begin{frame}{}
  \begin{center} \LARGE{Statistics - I} \end{center}
\end{frame}

\setbeamertemplate{background}{
  \parbox[c][\paperheight]{\paperwidth}{
    \hfill \pgfimage[height=\paperheight]{statistician}
  }
}
\begin{frame}{Disclaimer}
  \begin{columns}[t]
  \column{.5\textwidth}
  \begin{center}
  \Large{I am not a statistician}
  \end{center}
  \column{.5\textwidth}
  \end{columns}  
\end{frame}

\subsection{Random Sampling}

\setbeamertemplate{background}{
  \parbox[c][\paperheight]{\paperwidth}{
    \pgfimage[height=\paperheight]{population} \hfill
  }
}
\begin{frame}{Population}
  \begin{block}{}
  \begin{center}
  All possible users of your system
  \end{center}
  \end{block}
  \parbox[c][0.4\paperheight]{\paperwidth}{ }
\end{frame}

\setbeamertemplate{background}{
  \parbox[c][\paperheight]{\paperwidth}{
    \pgfimage[height=\paperheight]{sample} \hfill
  }
}
\begin{frame}{Sample}
  \begin{block}{}
  \begin{center}
  Representative subset of the population
  \end{center}
  \end{block}
  \parbox[c][0.4\paperheight]{\paperwidth}{ }
\end{frame}

\setbeamertemplate{background}{
  \parbox[c][\paperheight]{\paperwidth}{
    \pgfimage[height=\paperheight]{confounding} \hfill
  }
}
\begin{frame}{Bad sample}
  \begin{block}{}
  \begin{center}
  Sometimes it's not
  \end{center}
  \end{block}
  \parbox[c][0.4\paperheight]{\paperwidth}{ }
\end{frame}
\setbeamertemplate{background}{}

\begin{frame}{How to randomize?}
  \begin{itemize}
  \item Pick 10\% of users at random and always test them
  \end{itemize}
  \begin{center} \large{OR} \end{center}
  \begin{itemize}
  \item For each user, decide at random if they should be tested
  \end{itemize}
  \href{http://tech.bluesmoon.info/2010/01/statistics-of-performance-measurement.html}{\scriptsize{http://tech.bluesmoon.info/2010/01/statistics-of-performance-measurement.html}}
\end{frame}

\begin{frame}[fragile]{Select 10\% of users - I}
\begin{verbatim}
   if($sessionid % 10 === 0) {
      // instrument code for measurement
   }
\end{verbatim}
  \begin{itemize}
  \item Once a user enters the measurement bucket, they stay there until they log out
  \item Fixed set of users, so tests may be more consistent
  \item Error in the sample results in positive feedback
  \end{itemize}
\end{frame}

\begin{frame}[fragile]{Select 10\% of users - II}
\begin{verbatim}
   if(rand() < 0.1 * getrandmax()) {
      // instrument code for measurement
   }
\end{verbatim}
  \begin{itemize}
  \item For every request, a user has a 10\% chance of being tested
  \item Gets rid of positive feedback errors, but sample size \texttt{!=} 10\% of population
  \end{itemize}
\end{frame}

\begin{frame}{How big a sample is representative?}
  \begin{center}
  Select \(n\) such that \\
  \medskip
  \Large{\( \left| 1.96\frac{\sigma}{\sqrt{n}} \right| \le 5\% \mu \)}
  \end{center}
\end{frame}

\subsection{Margin of Error}

\setbeamertemplate{background}{
  \parbox[c][\paperheight]{\paperwidth}{
    \pgfimage[width=\paperwidth]{Standard_deviation_diagram} \hfill
  }
}
\begin{frame}{Standard Deviation}
  \begin{itemize}
  \item Standard deviation tells you the spread of the curve
  \item The narrower the curve, the more confident you can be
  \end{itemize}
  \parbox[c][0.58\paperheight]{\paperwidth}{ }
\end{frame}
\setbeamertemplate{background}{}


\begin{frame}{MoE at 95\% confidence}
  \begin{center}
  \Large{\( \pm 1.96\frac{\sigma}{\sqrt{n}} \)}
  \end{center}
\end{frame}

\begin{frame}{MoE \& Sample size}
  \begin{center}
  There is an inverse square root correlation between sample size and margin of error
  \end{center}
\end{frame}

\begin{frame}{}
  \begin{itemize}
  \item But wait... it's not complicated enough.
    \begin{itemize}
    \item<2-> We have different types of margins of error
    \item<3-> ...more about that later
    \end{itemize}
  \end{itemize}
\end{frame}

\subsection{Central Tendency}

\setbeamertemplate{background}{
  \parbox[c][\paperheight]{\paperwidth}{
    \pgfimage[height=\paperheight]{Normal_Distribution_PDF} \hfill
  }
}
\begin{frame}{Ding dong}
% Bell curve from wikipedia
\end{frame}
\setbeamertemplate{background}{}

\begin{frame}{One number}
  \begin{itemize}
  \item Mean (Arithmetic)
    \begin{itemize}
    \item Good for symmetric curves
    \item Affected by outliers
    \end{itemize}
  \end{itemize}
  \begin{center}
  \Large{ Mean(10, 11, 12, 11, 109) = 30 }
  \end{center}
\end{frame}

\begin{frame}{One number}
  \begin{itemize}
  \item Median
    \begin{itemize}
    \item Middle value measures central tendency well
    \item Not trivial to pull out of a DB
    \end{itemize}
  \end{itemize}
  \begin{center}
  \Large{ Median(10, 11, 12, 11, 109) = 11 }
  \end{center}
\end{frame}

\begin{frame}{One number}
  \begin{itemize}
  \item Mode
    \begin{itemize}
    \item Not often used
    \item Multi-modal distributions suggest problems
    \end{itemize}
  \end{itemize}
  \begin{center}
  \Large{ Mode(10, 11, 12, 11, 109) = 11 }
  \end{center}
\end{frame}

\begin{frame}{Other numbers}
  \begin{itemize}
  \item A percentile point in the distribution: 95\textsuperscript{th}, 98.5\textsuperscript{th} or 99\textsuperscript{th}
    \begin{itemize}
    \item Used to find out the worst user experience
    \item Makes more sense if you filter data first
    \end{itemize}
  \end{itemize}
  \begin{center}
  \Large{ P95\textsuperscript{th}(10, 11, 12, 11, 109) = 12 }
  \end{center}
\end{frame}

\begin{frame}{Other means}
  \begin{itemize}
  \item Geometric mean
    \begin{itemize}
    \item Good if your data is exponential in nature \\ (with the tail on the right)
    \end{itemize}
  \end{itemize}
  \begin{center}
  \Large{ GMean(10, 11, 12, 11, 109) = 16.68 }
  \end{center}
\end{frame}

\begin{frame}{Wait... how did I get that?}
  \pause

  \begin{center}
  \Large{\( \sqrt[N]{\Pi^N_{i=1}x_i} \)}
  \pause
  --- could lead to overflow \\
  \medskip
  \pause
  \medskip
  \Large{ \( e^{ \left(\frac{\Sigma^N_{i=1} log_e(x_i)}{N}\right) } \)} 
  --- computationally simpler
  \end{center}
\end{frame}

\begin{frame}{Other means}
  \begin{center}
  And there is also the Harmonic mean, but forget about that
  \end{center}
\end{frame}

\begin{frame}{...though consequently}
We have other margins of error
  \begin{itemize}
  \item Geometric margin of error
    \begin{itemize}
    \item Uses geometric standard deviation
    \end{itemize}
  \item Median margin of error
    \begin{itemize}
    \item Uses ranges of actual values from data set
    \end{itemize}
  \item<2-> Stick to the arithmetic MoE \\ -- simpler to calculate, simpler to read and not incorrect
  \end{itemize}
\end{frame}

\section{Statistics - II}

\begin{frame}{}
  \begin{center} \LARGE{Statistics - II} \end{center}
\end{frame}

\subsection{Filtering}

\setbeamertemplate{background}{
  \parbox[c][\paperheight]{\paperwidth}{
    \pgfimage[height=\paperheight]{outlier} \hfill
  }
}
\begin{frame}{Outliers}
  \begin{columns}[t]
  \column{.5\textwidth}
  \column{.5\textwidth}
  \begin{itemize}
  \item<2-> Out of range data points
  \item<3-> Nothing you can fix here
  \item<4-> There's even a book about them
  \end{itemize}
  \end{columns}
\end{frame}
\setbeamertemplate{background}{}

\begin{frame}{DNS problems can cause outliers}
  \begin{itemize}
  \item 2 or 3 DNS servers for an ISP
  \item 30 second timeout if first fails
  \item ... 30 second increase in page load time
  \item Maybe measure both and fix what you can
  \item \href{http://nms.lcs.mit.edu/papers/dns-ton2002.pdf}{\scriptsize{\texttt{http://nms.lcs.mit.edu/papers/dns-ton2002.pdf}}}
  \end{itemize}
\end{frame}

\setbeamertemplate{background}{
  \parbox[c][\paperheight]{\paperwidth}{
    \pgfimage[width=\paperwidth]{bandpass} \hfill
  }
}
\begin{frame}{Band-pass filtering}
  \only<2->{
  \begin{block}{}
  \begin{itemize}
  \item Strip everything outside a reasonable range
    \begin{itemize}
    \item Bandwidth range: 4kbps - 4Gbps
    \item Page load time: 50ms - 120s
    \end{itemize}
  \item You may need to relook at the ranges all the time
  \end{itemize}
  \end{block}
  }
\end{frame} 

\setbeamertemplate{background}{
  \parbox[c][\paperheight]{\paperwidth}{
    \pgfimage[height=\paperheight]{Boxplot_vs_PDF} \hfill
  }
}
\begin{frame}{IQR filtering}
  \only<2->{
  \begin{block}{}
  \begin{center}
  Here, we derive the range from the data
  \end{center}
  \end{block}
  }
\end{frame}
\setbeamertemplate{background}{}

\subsection{The Log-Normal distribution}

\begin{frame}{}
\begin{block}{}
\begin{center}
Let's look at some real charts
\end{center}
\end{block}
\end{frame}

\setbeamertemplate{background}{
  \parbox[c][\paperheight]{\paperwidth}{
    \vfill
    \pgfimage[width=\paperwidth]{linear_distribution}
    \vfill
  }
}
\begin{frame}{Bandwidth distribution for web devs}
  \parbox[c][0.55\paperheight]{\paperwidth}{ }
  \begin{center} x-axis is linear \end{center}
\end{frame}

\setbeamertemplate{background}{
  \parbox[c][\paperheight]{\paperwidth}{
    \pgfimage[height=\paperheight]{connection_x_overall} \hfill
  }
}
\begin{frame}{Now let's use log(kbps) instead of kbps}
  \parbox[c][0.55\paperheight]{\paperwidth}{ }
  \begin{center} x-axis is exponential \end{center}
\end{frame}
\setbeamertemplate{background}{}

\begin{frame}{Exponential == Geometric}
  \begin{itemize}
  \item Categories/Buckets grow exponentially
  \item Data is related geometrically
  \item<2-> Use the geometric mean and geometric margin of error
    \begin{itemize}
    \item \(Error\_range = \left[^{gmean}/_{gmoe}, gmean*gmoe\right]\)
    \end{itemize}
  \item<3-> Non-linear ranges are hard for humans to grok
  \end{itemize}
\end{frame}

\section{}

\begin{frame}{}
  \begin{center} \LARGE{So...} \end{center}
\end{frame}

\subsection{}
\begin{frame}{Further reading}
  \small{
  \begin{itemize}
  \item Web Performance - Not a Simple Number \\ \href{http://www.netforecast.com/Articles/BCR+C25+Web+Performance+-+Not+A+Simple+Number.pdf}{\tiny{http://www.netforecast.com/Articles/BCR+C25+Web+Performance+-+Not+A+Simple+Number.pdf}}
  \item Revisiting statistics for web performance (introduction to Log-Normal) \\ \href{http://home.pacbell.net/ciemo/statistics/WhatDoYouMean.pdf}{\tiny{http://home.pacbell.net/ciemo/statistics/WhatDoYouMean.pdf}}
  \item Random Sampling \\ \href{http://tech.bluesmoon.info/2010/01/statistics-of-performance-measurement.html}{\tiny{http://tech.bluesmoon.info/2010/01/statistics-of-performance-measurement.html}}
  \item Khan Academy's tutorials on statistics \\ \href{http://khanacademy.com/}{\tiny{http://khanacademy.com/}}
  \item Learning about Statistical Learning \\ \href{http://measuringmeasures.blogspot.com/2010/01/learning-about-statistical-learning.html}{\tiny{http://measuringmeasures.blogspot.com/2010/01/learning-about-statistical-learning.html}}
  \item Wikipedia articles on \href{http://en.wikipedia.org/wiki/Random_sample}{\color{blue}{\underline{Random Sampling}}}, \href{http://en.wikipedia.org/wiki/Central_tendency}{\color{blue}{\underline{Central Tendency}}}, \href{http://en.wikipedia.org/wiki/Standard_error_\%28statistics\%29}{\color{blue}{\underline{Standard Error}}}, \href{http://en.wikipedia.org/wiki/Confounding}{\color{blue}{\underline{Confounding}}}, \href{http://en.wikipedia.org/wiki/Mean}{\color{blue}{\underline{Means}}} and \href{http://en.wikipedia.org/wiki/IQR}{\color{blue}{\underline{IQR}}}
  \end{itemize}
  }
\end{frame}

\begin{frame}{Summary}
  \begin{itemize}
  \item Choose a reasonable sample size and sampling factor
  \item Tune sample size for minimal margin of error
  \item Decide based on your data whether to use mode, median or one of the means
  \item Figure out whether your data is Normal, Log-Normal or something else
  \item Filter out anomalous outliers
  \end{itemize}
\end{frame}

\setbeamertemplate{background}{
  \parbox[c][\paperheight]{\paperwidth}{
    \hfill \pgfimage[height=\paperheight]{../bluesmoon}
  }
}
\begin{frame}{\textit{contact me}}
  \begin{itemize}
  \item Philip Tellis
  \item \small{\texttt{philip@bluesmoon.info}}
  \item \href{http://bluesmoon.info/}{bluesmoon.info}
  \item @bluesmoon
  \end{itemize}
\end{frame}
\setbeamertemplate{background}{}

\begin{frame}{Photo credits}
  \small{
  \begin{itemize}
  \item \href{http://www.flickr.com/photos/leoffreitas/332360959/}{http://www.flickr.com/photos/leoffreitas/332360959/} by leoffreitas
  \item \href{http://www.flickr.com/photos/cobalt/56500295/}{http://www.flickr.com/photos/cobalt/56500295/} by cobalt123
  \item \href{http://www.flickr.com/photos/sophistechate/4264466015/}{http://www.flickr.com/photos/sophistechate/4264466015/} by Lisa Brewster
  \item \href{http://www.flickr.com/photos/nchoz/243216008/}{http://www.flickr.com/photos/nchoz/243216008/} by nchoz
  \end{itemize}
  }
\end{frame}

\begin{frame}{List of figures}
  \small{
  \begin{itemize}
  \item \href{http://en.wikipedia.org/wiki/File:Standard_deviation_diagram.svg}{http://en.wikipedia.org/wiki/File:Standard\_deviation\_diagram.svg}
  \item \href{http://en.wikipedia.org/wiki/File:Normal_Distribution_PDF.svg}{http://en.wikipedia.org/wiki/File:Normal\_Distribution\_PDF.svg}
  \item \href{http://en.wikipedia.org/wiki/File:KilroySchematic.svg}{http://en.wikipedia.org/wiki/File:KilroySchematic.svg}
  \item \href{http://en.wikipedia.org/wiki/File:Boxplot_vs_PDF.png}{http://en.wikipedia.org/wiki/File:Boxplot\_vs\_PDF.png}
  \end{itemize}
  }
\end{frame}

\end{document}


