\documentclass{beamer}

\mode<presentation>
{
  \usetheme{Copenhagen}

  \setbeamercovered{transparent}

}

\beamertemplatenavigationsymbolsempty

\definecolor{f2elblue}{RGB}{97,97,97}%{49,72,87}
\definecolor{dark-red}{RGB}{128,0,0}
\definecolor{dark-green}{RGB}{40,150,21}
\definecolor{light-gray}{gray}{0.85}
\definecolor{med-gray}{gray}{0.75}

\setbeamercolor{title in head/foot}{bg=orange!90!black,fg=white}
\setbeamercolor{author in head/foot}{bg=black,fg=white}
\setbeamercolor{subsection in head/foot}{bg=black!95,fg=white}
\setbeamercolor{structure}{fg=black}

\setbeamerfont{frametitle}{family=\rmfamily,shape=\itshape,size={\fontsize{14}{12}}}

\setbeamertemplate{blocks}[rounded][shadow=false]
\setbeamertemplate{items}[circle]

\setbeamertemplate{footline}%
{%
  \leavevmode%
  \hbox{%
    \begin{beamercolorbox}[wd=.45\paperwidth,ht=2.9ex,dp=1.125ex,leftskip=.3cm plus1fill,rightskip=.3cm]{author in head/foot}%
      \usebeamerfont{author in head/foot}\insertdate%
    \end{beamercolorbox}%
    \begin{beamercolorbox}[wd=.51\paperwidth,ht=2.9ex,dp=1.125ex,leftskip=.3cm,rightskip=.3cm plus1fil]{title in head/foot}%
      \usebeamerfont{title in head/foot}\insertshorttitle%
    \end{beamercolorbox}}%
    \begin{beamercolorbox}[wd=.04\paperwidth,ht=2.9ex,dp=1.125ex,center,leftskip=3pt plus1fill,rightskip=4pt]{author in head/foot}%
      \usebeamerfont{author in head/foot}\insertframenumber%
    \end{beamercolorbox}%
  \vskip0pt%
}%


\usepackage[english]{babel}

\usepackage[latin1]{inputenc}

\usepackage{times}
\usepackage[T1]{fontenc}

\usepackage{ulem}

\usepackage{mathtools}

\author{Philip Tellis / \texttt{philip@bluesmoon.info}}

\title{The Statistics of Web Performance Analysis}

\date{\href{http://www.meetup.com/Web-Performance-NY/events/78574902/}{New York \#WebPerf Meetup} / 2012-11-29}


\pgfdeclareimage[height=0.5cm]{bluesmoon-logo}{../ui/default/bodybg}
\pgfdeclareimage[height=11pt]{cc-licence}{../by-nc-sa-3.0-88x31}
\logo{\pgfuseimage{cc-licence}}


%%%% Begin defining new commands for this template

% Serial number in slide title
\newcommand{\sn}[1]{\textrm{\textit{\Huge{\raisebox{-3pt}[4pt][8pt]{\textcolor{f2elblue}{#1}}}}}\hspace{4pt}}
% large Centered Italic Roman text within a slide (building block for the following)
\newcommand{\innersplash}[1]{
  \begin{center}
    \Large \textrm{\textit{ #1 } }
  \end{center}
}
% Slide that has just one large centered thing to say
\newcommand{\splashslide}[2][{}]{
  \begin{frame}
  \frametitle{#1}
  \innersplash{#2}
  \end{frame}
}
% Leadin slide at the start of an exploit
\newcommand{\leadinslide}[2]{
  \splashslide{
     {\fontsize{150}{20}\selectfont{\raisebox{0pt}[90pt][0pt]{\textcolor{light-gray}{#1}}}} \\ \huge \textcolor{gray}{#2}
  }
}
% Suggested fix at the end of an exploit
\newcommand{\suggestedfix}[2]{
  \splashslide[\sn{#1} Suggested fix]{{#2}}
}
% Syntax highlighting
\def\gray<#1>#2{\textcolor<#1>{gray}{\textbf<#1>{#2}}}
\def\brown<#1>#2{\textcolor<#1>{brown}{\textbf<#1>{#2}}}
\def\green<#1>#2{\textcolor<#1>{dark-green}{\textbf<#1>{#2}}}
\def\blue<#1>#2{\textcolor<#1>{blue}{\textbf<#1>{#2}}}
\def\purple<#1>#2{\textcolor<#1>{purple}{\textbf<#1>{#2}}}
\def\red<#1>#2{\textcolor<#1>{red}{\textbf<#1>{#2}}}

%%%% End of new commands




\begin{document}

\setbeamertemplate{background}{
  \parbox[c][\paperheight]{\paperwidth}{
    \vfill \hfill \pgfimage[height=\paperheight]{../bluesmoon}
  }
}
\begin{frame}
  \begin{itemize}
  \item Philip Tellis
  \item \href{http://www.soasta.com/}{www.soasta.com}
  \item \small{\texttt{philip@bluesmoon.info}}
  \item \href{http://twitter.com/bluesmoon}{@bluesmoon}
  \item geek paranoid speedfreak
  \item \href{http://bluesmoon.info/}{http://bluesmoon.info/}
  \end{itemize}
\end{frame}
\setbeamertemplate{background}{}

\splashslide{I'm a Web Speedfreak}

\splashslide{\pgfimage[width=.8\paperwidth]{lognormal} \\ We measure real user website performance}

\splashslide{This talk is about the Statistics we learned while building it}

\begin{frame}
  \titlepage
\end{frame}

\leadinslide{0}{Numbers}

\splashslide{Accurately measure page performance$^{*}$}

\splashslide[Be unintrusive]{If you try to measure something accurately, you will change something related \\ \hfill \tiny{-- Heisenberg's uncertainty principle}}

\splashslide{And one number to rule them all}

\begin{frame}{What do we measure?}
  \begin{itemize}
  \item Network Throughput
  \item Network Latency
  \item User perceived page load time
  \item Everything in NavTiming
  \item DOM \& Memory use
  \end{itemize}
\end{frame}

\splashslide{We measure real user data}

\splashslide{Which is noisy}

\leadinslide{1}{Statistics - 1}

\setbeamertemplate{background}{
  \parbox[c][\paperheight]{\paperwidth}{
    \hfill \pgfimage[height=\paperheight]{statistician}
  }
}
\begin{frame}{Disclaimer}
  \begin{columns}[t]
  \column{.5\textwidth}
  \begin{center}
  \Large{I am not a statistician}
  \end{center}
  \column{.5\textwidth}
  \end{columns}  
\end{frame}
\setbeamertemplate{background}{}

\leadinslide{1-1}{Random Sampling}

\setbeamertemplate{background}{
  \parbox[c][\paperheight]{\paperwidth}{
    \pgfimage[height=\paperheight]{population} \hfill
  }
}
\begin{frame}{Population}
  \begin{block}{}
  \begin{center}
  All possible users of your system
  \end{center}
  \end{block}
  \parbox[c][0.4\paperheight]{\paperwidth}{ }
\end{frame}

\setbeamertemplate{background}{
  \parbox[c][\paperheight]{\paperwidth}{
    \pgfimage[height=\paperheight]{sample} \hfill
  }
}
\begin{frame}{Sample}
  \begin{block}{}
  \begin{center}
  Representative subset of the population
  \end{center}
  \end{block}
  \parbox[c][0.4\paperheight]{\paperwidth}{ }
\end{frame}

\setbeamertemplate{background}{
  \parbox[c][\paperheight]{\paperwidth}{
    \pgfimage[height=\paperheight]{confounding} \hfill
  }
}
\begin{frame}{Bad sample}
  \begin{block}{}
  \begin{center}
  Sometimes it's not
  \end{center}
  \end{block}
  \parbox[c][0.4\paperheight]{\paperwidth}{ }
\end{frame}

\setbeamertemplate{background}{
  \parbox[c][0.9\paperheight]{0.85\paperwidth}{
    \hfill \pgfimage[width=0.7\paperwidth]{random_number} 
  }
}
\begin{frame}{How to randomize?}
\parbox[c][1.2\paperheight]{\paperwidth}{}
\hfill \tiny{\href{http://xkcd.com/221/}{http://xkcd.com/221/}}
\end{frame}
\setbeamertemplate{background}{}

\begin{frame}{How to randomize?}
  \begin{itemize}
  \item Pick 10\% of users at random and always test them
  \end{itemize}
  \begin{center} \large{OR} \end{center}
  \begin{itemize}
  \item For each user, decide at random if they should be tested
  \end{itemize}
  \href{http://tech.bluesmoon.info/2010/01/statistics-of-performance-measurement.html}{\scriptsize{http://tech.bluesmoon.info/2010/01/statistics-of-performance-measurement.html}}
\end{frame}

\begin{frame}[fragile]{Select 10\% of users - I}
\begin{semiverbatim}
   \green<1>{if}(\$sessionid \% \brown<1>{10} === \brown<1>{0}) \{
      // instrument code for measurement
   \}
\end{semiverbatim}
  \begin{itemize}
  \item Once a user enters the measurement bucket, they stay there until they log out
  \item Fixed set of users, so tests may be more consistent
  \item Error in the sample results in positive feedback
  \end{itemize}
\end{frame}

\begin{frame}[fragile]{Select 10\% of users - II}
\begin{semiverbatim}
   \green<1>{if}(\blue<1>{rand}() < \brown<1>{0.1} * \blue<1>{getrandmax}()) \{
      // instrument code for measurement
   \}
\end{semiverbatim}
  \begin{itemize}
  \item For every request, a user has a 10\% chance of being tested
  \item Gets rid of positive feedback errors, but sample size \texttt{!=} 10\% of population
  \end{itemize}
\end{frame}

\splashslide[How big a sample is representative?]{Select \(n\) such that \\ \medskip \LARGE \( \mathrm{\left| 1.96\frac{\sigma}{\sqrt{n}} \right| \le 5\% \mu} \) \\ \tiny Not the best measure, but fast and easy to calculate}

\leadinslide{1-2}{Margin of Error}

\setbeamertemplate{background}{
  \parbox[c][\paperheight]{\paperwidth}{
    \pgfimage[width=\paperwidth]{Standard_deviation_diagram} \hfill
  }
}
\begin{frame}{Standard Deviation}
  \begin{itemize}
  \item Standard deviation tells you the spread of the curve
  \item The narrower the curve, the more confident you can be
  \end{itemize}
  \parbox[c][0.58\paperheight]{\paperwidth}{ }
\end{frame}
\setbeamertemplate{background}{}


\splashslide[MoE at 95\% confidence]{ \Huge \( \mathrm{\pm 1.96\frac{\sigma}{\sqrt{n}}} \) }

\splashslide[MoE \& Sample size]{There is an inverse square root correlation between sample size and margin of error}

\leadinslide{1-3}{Central Tendency}

\setbeamertemplate{background}{
  \parbox[c][0.9\paperheight]{\paperwidth}{
    \pgfimage[height=\paperheight]{Normal_Distribution_PDF} \hfill
  }
}
\begin{frame}
% Bell curve from wikipedia
\end{frame}
\setbeamertemplate{background}{}

\begin{frame}{One number}
  \begin{itemize}
  \item Mean (Arithmetic)
    \begin{itemize}
    \item Good for symmetric curves
    \item Affected by outliers
    \end{itemize}
  \end{itemize}
  \begin{center}
  \Large{ Mean(10, 11, 12, 11, 109) = 30 }
  \end{center}
\end{frame}

\begin{frame}{One number}
  \begin{itemize}
  \item Median
    \begin{itemize}
    \item Middle value measures central tendency well
    \item Not trivial to pull out of a DB
    \end{itemize}
  \end{itemize}
  \begin{center}
  \Large{ Median(10, 11, 12, 11, 109) = 11 }
  \end{center}
\end{frame}

\begin{frame}{One number}
  \begin{itemize}
  \item Mode
    \begin{itemize}
    \item Not often used
    \item Multi-modal distributions suggest problems
    \end{itemize}
  \end{itemize}
  \begin{center}
  \Large{ Mode(10, 11, 12, 11, 109) = 11 }
  \end{center}
\end{frame}

\begin{frame}{Other numbers}
  \begin{itemize}
  \item A percentile point in the distribution: 95\textsuperscript{th}, 98.5\textsuperscript{th} or 99\textsuperscript{th}
    \begin{itemize}
    \item Used to find out the worst user experience
    \item Makes more sense if you filter data first
    \end{itemize}
  \end{itemize}
  \begin{center}
  \Large{ P95\textsuperscript{th}(10, 11, 12, 11, 109) = 12 }
  \end{center}
\end{frame}

\begin{frame}{Other means}
  \begin{itemize}
  \item Geometric mean
    \begin{itemize}
    \item Good if your data is exponential in nature \\ (with the tail on the right)
    \end{itemize}
  \end{itemize}
  \begin{center}
  \Large{ GMean(10, 11, 12, 11, 109) = 16.68 }
  \end{center}
\end{frame}

\begin{frame}{Wait... how did I get that?}
  \pause

  \begin{center}
  \Large{\( \sqrt[N]{\Pi^N_{i=1}x_i} \)}
  \pause
  --- could lead to overflow \\
  \medskip
  \pause
  \medskip
  \Large{ \( e^{ \left(\frac{\Sigma^N_{i=1} log_e(x_i)}{N}\right) } \)} 
  --- computationally simpler
  \end{center}
\end{frame}

\begin{frame}{Other means}
  \begin{center}
  And there is also the Harmonic mean, but forget about that
  \end{center}
\end{frame}

\begin{frame}{...though consequently}
We have other margins of error
  \begin{itemize}
  \item Geometric margin of error
    \begin{itemize}
    \item Uses geometric standard deviation
    \end{itemize}
  \item Median margin of error
    \begin{itemize}
    \item Uses ranges of actual values from data set
    \end{itemize}
  \item<2-> Stick to the arithmetic MoE \\ -- simpler to calculate, simpler to read and not incorrect
  \end{itemize}
\end{frame}

\leadinslide{2}{Statistics - 2}

\leadinslide{2-1}{Distributions}

\splashslide{Let's look at some real charts}

\setbeamertemplate{background}{
  \parbox[c][\paperheight]{\paperwidth}{
    \vfill
    \pgfimage[width=\paperwidth]{sparse-distribution}
    \vfill
  }
}
\begin{frame}{Sparse Distribution}
\end{frame}

\setbeamertemplate{background}{
  \parbox[c][\paperheight]{\paperwidth}{
    \pgfimage[width=\paperwidth]{lognormal-distribution} \hfill
  }
}
\begin{frame}{Log-normal distribution}
\end{frame}

\setbeamertemplate{background}{
  \parbox[c][\paperheight]{\paperwidth}{
    \pgfimage[width=\paperwidth]{bimodal-distribution} \hfill
  }
}
\begin{frame}{Bimodal distribution}
\end{frame}
\setbeamertemplate{background}{}

\splashslide{What does all of this mean?}

\begin{frame}{Distributions}
\begin{itemize}
  \item Sparse distribution suggests that you don't have enough data points
  \item Log-normal distribution is typical
  \item Bi-modal distribution suggests two (or more) distributions combined
\end{itemize}
\end{frame}

\splashslide{In practice, a bi-modal distribution is not uncommon}

\splashslide{Hint: Does your site do a lot of back-end caching?}

\leadinslide{2-2}{Filtering}

\setbeamertemplate{background}{
  \parbox[c][\paperheight]{\paperwidth}{
    \pgfimage[height=\paperheight]{outlier} \hfill
  }
}
\begin{frame}{Outliers}
  \begin{columns}[t]
  \column{.5\textwidth}
  \column{.5\textwidth}
  \begin{itemize}
  \item<2-> Out of range data points
  \item<3-> Nothing you can fix here
  \item<4-> There's even a book about them
  \end{itemize}
  \end{columns}
\end{frame}
\setbeamertemplate{background}{}

\begin{frame}{DNS problems can cause outliers}
  \begin{itemize}
  \item 2 or 3 DNS servers for an ISP
  \item 30 second timeout if first fails
  \item ... 30 second increase in page load time
  \item Maybe measure both and fix what you can
  \item \href{http://nms.lcs.mit.edu/papers/dns-ton2002.pdf}{\scriptsize{\texttt{http://nms.lcs.mit.edu/papers/dns-ton2002.pdf}}}
  \end{itemize}
\end{frame}

\setbeamertemplate{background}{
  \parbox[c][\paperheight]{\paperwidth}{
    \pgfimage[width=\paperwidth]{bandpass} \hfill
  }
}
\begin{frame}{Band-pass filtering}
  \only<2->{
  \begin{block}{}
  \begin{itemize}
  \item Strip everything outside a reasonable range
    \begin{itemize}
    \item Bandwidth range: 4kbps - 4Gbps
    \item Page load time: 50ms - 120s
    \end{itemize}
  \item You may need to relook at the ranges all the time
  \end{itemize}
  \end{block}
  }
\end{frame} 

\setbeamertemplate{background}{
  \parbox[c][\paperheight]{\paperwidth}{
    \pgfimage[height=\paperheight]{Boxplot_vs_PDF} \hfill
  }
}
\begin{frame}{IQR filtering}
  \only<2->{
  \begin{block}{}
  \begin{center}
  Here, we derive the range from the data
  \end{center}
  \end{block}
  }
\end{frame}
\setbeamertemplate{background}{}

\leadinslide{2-3}{Smoothing}

\splashslide{How does your site compare with yesterday, last week and last month?}

\splashslide{Are certain spikes normal or anomalous?}

\begin{frame}{Simple Moving Average}
  \begin{center}
  \Large \textrm{\textit{Each point is the average of the last k points}}

  \vspace{18pt}

  \LARGE
  \( s_t = \frac{1}{k} \Sigma^{k-1}_{n=0}x_{t-n} \)

  \vspace{12pt}

  \(     = s_{t-1} + \frac{x_t - x_{t-k}}{k} \)

  \vspace{18pt}
  \Large \textrm{\textit{Average lags behind any trend}}
  \end{center}
\end{frame}

\splashslide{\href{http://en.wikipedia.org/wiki/Exponential\_smoothing\#Double\_exponential\_smoothing}{Holt-Winters Double Exponential Smoothing \\ \tiny wikipedia.org/wiki/Exponential\_smoothing\#Double\_exponential\_smoothing}}

\begin{frame}{Holt-Winters Smoothing in Graphite}
  \pgfimage[width=0.85\paperwidth]{lonely-planet-holt-winters}
  \vspace{12pt}
  Thanks to the smart folks at Lonely Planet for this chart: \\
  \href{http://velocityconf.com/velocityeu2012/public/schedule/detail/26634}{velocityconf.com/velocityeu2012/public/schedule/detail/26634}
\end{frame}

\leadinslide{--}{OK Stop}

\splashslide[Further Reading]{\href{http://www.lognormal.com/blog/2012/08/13/analysing-performance-data/}{lognormal.com/blog/2012/08/13/analysing-performance-data/}}

\begin{frame}{Summary}
  \begin{itemize}
  \item Choose a reasonable sample size and sampling factor
  \item Tune sample size for minimal margin of error
  \item Decide based on your data whether to use mode, median or one of the means
  \item Figure out whether your data is Normal, Log-Normal or something else
  \item Filter out anomalous outliers
  \item Use Holt-Winters smoothing to scan for anomalous behaviour
  \end{itemize}
\end{frame}


\setbeamertemplate{background}{
  \parbox[c][\paperheight]{\paperwidth}{
    \vfill \hfill \pgfimage[height=\paperheight]{../bluesmoon}
  }
}
\begin{frame}
  \begin{itemize}
  \item Philip Tellis
  \item \href{http://www.lognormal.com/}{\pgfimage[width=.2\paperwidth]{lognormal}\textrm{\textcolor{med-gray}{.com}}}
  \item \small{\texttt{philip@bluesmoon.info}}
  \item \href{http://twitter.com/bluesmoon}{@bluesmoon}
  \item geek paranoid speedfreak
  \item \href{http://bluesmoon.info/}{http://bluesmoon.info/}
  \end{itemize}
\end{frame}
\setbeamertemplate{background}{}

\splashslide{\Huge Thank you }

\begin{frame}{Photo credits}
  \small{
  \begin{itemize}
  \item \href{http://www.flickr.com/photos/leoffreitas/332360959/}{http://www.flickr.com/photos/leoffreitas/332360959/} by leoffreitas
  \item \href{http://www.flickr.com/photos/cobalt/56500295/}{http://www.flickr.com/photos/cobalt/56500295/} by cobalt123
  \item \href{http://www.flickr.com/photos/sophistechate/4264466015/}{http://www.flickr.com/photos/sophistechate/4264466015/} by Lisa Brewster
  \end{itemize}
  }
\end{frame}

\begin{frame}{List of figures}
  \small{
  \begin{itemize}
  \item \href{http://en.wikipedia.org/wiki/File:Standard_deviation_diagram.svg}{http://en.wikipedia.org/wiki/File:Standard\_deviation\_diagram.svg}
  \item \href{http://en.wikipedia.org/wiki/File:Normal_Distribution_PDF.svg}{http://en.wikipedia.org/wiki/File:Normal\_Distribution\_PDF.svg}
  \item \href{http://en.wikipedia.org/wiki/File:KilroySchematic.svg}{http://en.wikipedia.org/wiki/File:KilroySchematic.svg}
  \item \href{http://en.wikipedia.org/wiki/File:Boxplot_vs_PDF.png}{http://en.wikipedia.org/wiki/File:Boxplot\_vs\_PDF.png}
  \end{itemize}
  }
\end{frame}

\end{document}


