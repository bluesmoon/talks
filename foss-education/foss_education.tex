\documentclass{beamer}

\mode<presentation>
{
  \usetheme{Warsaw}

  \setbeamercovered{transparent}

}

\beamertemplatenavigationsymbolsempty

\setbeamercolor{title in head/foot}{bg=orange!90!black,fg=white}
\setbeamercolor{subsection in head/foot}{bg=black!95,fg=white}
\setbeamercolor{structure}{fg=blue!10!black}

\setbeamertemplate{blocks}[rounded][shadow=false]

\setbeamertemplate{footline}%
{%
  \leavevmode%
  \hbox{\begin{beamercolorbox}[wd=.5\paperwidth,ht=2.5ex,dp=1.125ex,leftskip=.3cm plus1fill,rightskip=.3cm]{author in head/foot}%
    \usebeamerfont{author in head/foot}\insertdate%
  \end{beamercolorbox}%
  \begin{beamercolorbox}[wd=.5\paperwidth,ht=2.5ex,dp=1.125ex,leftskip=.3cm,rightskip=.3cm plus1fil]{title in head/foot}%
    \usebeamerfont{title in head/foot}\insertshorttitle%
  \end{beamercolorbox}}%
  \vskip0pt%
}%

\usepackage[english]{babel}

\usepackage[latin1]{inputenc}

\usepackage{times}
\usepackage[T1]{fontenc}

\usepackage{ulem}

\author{Philip Tellis / \texttt{philip@bluesmoon.info}}

\institute{Yahoo!}

\title{Why Foss in Education makes sense}

\date{FOSS.IN/2005 -- 2005/12/01}


\pgfdeclareimage[height=0.5cm]{bluesmoon-logo}{../ui/default/bodybg}
\pgfdeclareimage[height=11pt]{cc-licence}{../by-nc-sa-3.0-88x31}
\logo{\pgfuseimage{cc-licence}}





\begin{document}

\begin{frame}
  \titlepage
\end{frame}


\setbeamertemplate{background}{
  \parbox[c][\paperheight]{\paperwidth}{
    \hfill \pgfimage[height=\paperheight]{../bluesmoon}
  }
}
\begin{frame}{\textit{\$ whoami}}
  \begin{itemize}
  \item Philip Tellis
  \item \small{\texttt{philip@bluesmoon.info}}
  \item \href{http://twitter.com/bluesmoon}{@bluesmoon}
  \item yahoo
  \item geek
  \end{itemize}
\end{frame}
\setbeamertemplate{background}{}

\begin{frame}{What this talk is not about}
   \begin{itemize}
   \item Educational tools under linux
   \item Free educational tools
   \item Cost benefits of using foss in schools
   \end{itemize}
\end{frame}

\begin{frame}{What this talk is about}
   \begin{itemize}
   \item Foss as a means of fostering education
   \item Foss as a means of validating research
   \item The \textit{fossification} of education
   \end{itemize}
\end{frame}

\begin{frame}{The role of \textit{computers} in education}
   \begin{itemize}
   \item Computer Education
   \item Other Education
   \item Academic related
   \item Administrative
   \end{itemize}
\end{frame}

\begin{frame}{The role of computers in \textit{education}}
   \begin{itemize}
   \item Instruction delivery
   \item Instruction enabling
   \item Administration
   \end{itemize}
\end{frame}

\begin{frame}{Instruction delivery}
   \begin{itemize}
   \item<1-> Throw information at students until they know it all
   \item<2-> Don't let them go ahead unless they pass all tests
   \item<3-> Throw data at students and let them process it
   \item<4-> Let the student derive information
   \end{itemize}
\end{frame}

\begin{frame}{Instruction enabling}
   \begin{itemize}
   \item The computer as the laboratory
   \item C programming can only be taught on a computer
   \item Playing with Math and Science on the computer
   \end{itemize}
\end{frame}

\begin{frame}
   \begin{block}{}
   \begin{center}
   Do we want the computer to program the child or the child to program the computer?
   \end{center}
   \end{block}
\end{frame}

\begin{frame}{Language learning is natural}
   \begin{itemize}
   \item<1-> Natural languages are learnt through living
   \item<2-> Learning is inherited from ones surroundings
   \item<3-> Create virtual lands where the mother tongue is the topic to be learnt
   \item<4-> Mathland, Physicsland, Poetryland
   \end{itemize}
\end{frame}

\begin{frame}{Debugging ones mistakes}
   \begin{itemize}
   \item Don't punish children when they make mistakes
   \item Teach them to debug
   \item Teach them to read other peoples' problems and debug them too
   \end{itemize}
\end{frame}

\begin{frame}{This is where Foss shines}
   \begin{block}{}
   \begin{center}
   FOSS is great for learning because the source code is available. Not just for reading, but for modification, and experimentation.
   \end{center}
   \end{block}
\end{frame}

\begin{frame}{Use FOSS tools instead of proprietary ones}
   \begin{itemize}
   \item Should we teach students specific tools or give them the ability to learn any tool?
   \item Should we NOT teach the current popular tools?
   \item Throw responsibility into the hands of students
   \end{itemize}
\end{frame}

\begin{frame}
   \begin{block}{}
   \begin{center}
   Popularity begets Obsolescence
   \end{center}
   \end{block}
\end{frame}

\begin{frame}{Academics and FOSS}
   \begin{itemize}
   \item Academia spreads knowledge by publishing papers, results and findings.
   \item Foss is a solid implementation of these ideas
   \item Foss allows one to build on another's knowledge
   \item Foss allows verifiability -- the basis of all scientific publishings
   \end{itemize}
\end{frame}

\begin{frame}{We need to foster this at the school level}
   \begin{itemize}
   \item Students of higher classes can build tools for lower classes
   \item Students work in a virtual world for the topic they study
   \item Programming expertise is not required, but domain expertise is built
   \item Students learn by collaborating and studying others' implementations
   \end{itemize}
\end{frame}

\begin{frame}{Doing, teaching, collaborating}
   \begin{itemize}
   \item Learning is fostered by doing, teaching and collaborating
   \item This is why Foss makes sense for education
   \item We have plenty of examples in computer science and bioinformatics
   \item Let's apply this to other fields too
   \end{itemize}
\end{frame}

\begin{frame}
   \begin{block}{}
   \begin{center}
   Education \alt<2->{wants}{needs} to be Free and Open
   \end{center}
   \end{block}
\end{frame}

\begin{frame}{Thanks}
   \begin{center}
   \small{\href{http://tech.bluesmoon.info/2005/11/why-foss-in-education-makes-sense.html}{http://tech.bluesmoon.info/2005/11/why-foss-in-education-makes-sense.html}}
   \end{center}
\end{frame}

\end{document}
