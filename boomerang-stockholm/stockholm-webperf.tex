\documentclass{beamer}

\mode<presentation>
{
  \usetheme{Copenhagen}

  \setbeamercovered{transparent}

}

\beamertemplatenavigationsymbolsempty

\definecolor{f2elblue}{RGB}{97,97,97}%{49,72,87}
\definecolor{dark-red}{RGB}{128,0,0}
\definecolor{dark-green}{RGB}{40,150,21}
\definecolor{light-gray}{gray}{0.85}
\definecolor{med-gray}{gray}{0.75}

\setbeamercolor{title in head/foot}{bg=orange!90!black,fg=white}
\setbeamercolor{author in head/foot}{bg=black,fg=white}
\setbeamercolor{subsection in head/foot}{bg=black!95,fg=white}
\setbeamercolor{structure}{fg=black}

\setbeamerfont{frametitle}{family=\rmfamily,shape=\itshape,size={\fontsize{14}{12}}}

\setbeamertemplate{blocks}[rounded][shadow=false]
\setbeamertemplate{items}[circle]

\setbeamertemplate{footline}%
{%
  \leavevmode%
  \hbox{%
    \begin{beamercolorbox}[wd=.45\paperwidth,ht=2.9ex,dp=1.125ex,leftskip=.3cm plus1fill,rightskip=.3cm]{author in head/foot}%
      \usebeamerfont{author in head/foot}\insertdate%
    \end{beamercolorbox}%
    \begin{beamercolorbox}[wd=.51\paperwidth,ht=2.9ex,dp=1.125ex,leftskip=.3cm,rightskip=.3cm plus1fil]{title in head/foot}%
      \usebeamerfont{title in head/foot}\insertshorttitle%
    \end{beamercolorbox}}%
    \begin{beamercolorbox}[wd=.04\paperwidth,ht=2.9ex,dp=1.125ex,center,leftskip=3pt plus1fill,rightskip=4pt]{author in head/foot}%
      \usebeamerfont{author in head/foot}\insertframenumber%
    \end{beamercolorbox}%
  \vskip0pt%
}%


\usepackage[english]{babel}

\usepackage[latin1]{inputenc}

\usepackage{times}
\usepackage[T1]{fontenc}

\usepackage{ulem}
\usepackage{mathtools}

\author{Philip Tellis / \texttt{ptellis@soasta.com}}

\title{Boomerang: Measuring Web Performance with JavaScript}

\date{\href{http://www.meetup.com/Stockholm-Web-Performance-Group/events/213557192/}{Stockholm \#WebPerf Meetup} / 2014-10-30}


\pgfdeclareimage[height=0.5cm]{bluesmoon-logo}{../ui/default/bodybg}
\pgfdeclareimage[height=11pt]{cc-licence}{../by-nc-sa-3.0-88x31}
\logo{\pgfuseimage{cc-licence}}


%%%% Begin defining new commands for this template

% Serial number in slide title
\newcommand{\sn}[1]{\textrm{\textit{\Huge{\raisebox{-3pt}[4pt][8pt]{\textcolor{f2elblue}{#1}}}}}\hspace{4pt}}

% large Centered Italic Roman text within a slide (building block for the following)
\newcommand{\innersplash}[1]{
  \begin{center}
    \large \textrm{\textit{ #1 } }
  \end{center}
}

% Slide that has just one large centered thing to say
\newcommand{\splashslide}[2][{}]{
  \begin{frame}
  \frametitle{#1}
  \innersplash{#2}
  \end{frame}
}

% Leadin slide at the start of an exploit
\newcommand{\leadinslide}[2]{
  \splashslide{
     {\fontsize{150}{20}\selectfont{\raisebox{0pt}[90pt][0pt]{\textcolor{light-gray}{#1}}}} \\ \huge \textcolor{gray}{#2}
  }
}

% Suggested fix at the end of an exploit
\newcommand{\suggestedfix}[2]{
  \splashslide[\sn{#1} Suggested fix]{{#2}}
}

%Splashslide with Background Image
\newcommand{\splashimage}[4][1]{
  \setbeamertemplate{background}{
    \parbox[c][{#1}\paperheight]{\paperwidth}{
      \vfill \hfill \pgfimage[width=\paperwidth]{#2}
    }
  }
  \begin{frame}{#3}
    \parbox[r][0.75\paperheight]{0.9\paperwidth}{
      \vfill {#4}
    }
  \end{frame}
  \setbeamertemplate{background}{}
}

%Splashslide with Vertical Background Image
\newcommand{\splashimagev}[4][1]{
  \setbeamertemplate{background}{
    \parbox[c][0.95\paperheight]{{#1}\paperwidth}{
      \vfill \hfill \pgfimage[height=0.86\paperheight]{#2}
    }
  }
  \begin{frame}{#3}
    \parbox[r][0.75\paperheight]{0.9\paperwidth}{
      \vfill {#4}
    }
  \end{frame}
  \setbeamertemplate{background}{}
}

% Syntax highlighting
\def\gray<#1>#2{\textcolor<#1>{gray}{\textbf<#1>{#2}}}
\def\brown<#1>#2{\textcolor<#1>{brown}{\textbf<#1>{#2}}}
\def\green<#1>#2{\textcolor<#1>{dark-green}{\textbf<#1>{#2}}}
\def\blue<#1>#2{\textcolor<#1>{blue}{\textbf<#1>{#2}}}
\def\purple<#1>#2{\textcolor<#1>{purple}{\textbf<#1>{#2}}}
\def\red<#1>#2{\textcolor<#1>{red}{\textbf<#1>{#2}}}

\newcommand{\textsubscript}[1]{\ensuremath{_{\textrm{#1}}}}
%%%% End of new commands




\begin{document}

\begin{frame}
  \titlepage
\end{frame}

\setbeamertemplate{background}{
  \parbox[c][\paperheight]{\paperwidth}{
    \vfill \hfill \pgfimage[height=\paperheight]{../bluesmoon}
  }
}
\begin{frame}
  \begin{itemize}
  \item Philip Tellis
  \item \href{http://twitter.com/bluesmoon}{@bluesmoon}
  \item \href{http://bluesmoon.info/}{ptellis@soasta.com}
  \item \href{http://www.soasta.com/}{SOASTA}
  \item \href{http://lognormal.github.com/boomerang/doc/}{boomerang}
  \end{itemize}
\end{frame}
\setbeamertemplate{background}{}

\splashslide{I \texttt{<3} JavaScript}

\splashslide{I'm a Speedfreak (the network kind)}

\splashimage[0.82]{wpt-mpulse}{SOASTA mPulse \& boomerang}{We measure real user website performance}

\splashslide{This talk is (mostly) about how we abuse JavaScript to do it}

\begin{frame}[fragile]
\frametitle{First, a note about the code}
\begin{center}
Note that in the code that follows,

\vspace{1.5em}
\texttt{+ \green<1>{new} \blue<1>{Date}}

\vspace{1.5em}
is equivalent to

\vspace{1.5em}
\texttt{\green<1>{new} \blue<1>{Date}().getTime()}
\end{center}
\end{frame}

\leadinslide{1}{Latency}

\begin{frame}{Sort in ascending order of signal latency}
\begin{itemize}
  \item Electrons through copper
  \item Light through fibre
  \item Pulsars
  \item Station Wagons
  \item Smoke Signals
\end{itemize}
\end{frame}

\begin{frame}{Sort in ascending order of signal latency}
\begin{enumerate}
  \item Pulsars (light through vacuum)
  \item Smoke Signals (light through air)
  \item Electrons through copper / Light through fibre
  \item Station Wagons (possibly highest bandwidth)
\end{enumerate}
\end{frame}

\splashslide[\sn{1}Blinking Lights]{ It takes about 16ms for light to get from SF to NYC \\ (24ms through fibre) \\ \vspace{1em} \only<1>{ \ldots } \only<2>{Though it takes about 100ms to ping... why?} }

\setbeamertemplate{background}{
  \parbox[c][.905\paperheight]{.93\paperwidth}{
    \vfill \hfill \pgfimage[height=.762\paperheight]{http-get}
  }
}
\begin{frame}{\sn{1}HTTP}
\end{frame}
\setbeamertemplate{background}{}

\splashslide{So to measure latency, we need to send 1 packet each way, and time it}

\begin{frame}[fragile]
\frametitle{\sn{1} Network latency in JavaScript}
\begin{semiverbatim}
  \green<1>{var} ts, rtt, img = \green<1>{new} \blue<1>{Image};
  img.onload=\green<1>{function}() \{ rtt=(+\green<1>{new} \blue<1>{Date} - ts) \};
  ts = +\green<1>{new} \blue<1>{Date};
  img.src="/1x1.gif";
\end{semiverbatim}
\end{frame}

\begin{frame}{\sn{1}Notes}
\begin{itemize}
  \item 1x1 gif is 35 bytes
  \item including HTTP headers, this is smaller than a TCP packet
  \item Fires onload on all browsers
  \item 0 byte image fires onerror
  \item which is indistinguishable from network error
\end{itemize}
\end{frame}

\leadinslide{2}{TCP handshake}

\setbeamertemplate{background}{
  \parbox[c][.9\paperheight]{.8\paperwidth}{
    \vfill \hfill \pgfimage[height=.8\paperheight]{tcp-handshake}
  }
}
\begin{frame}{\sn{2}ACK-ACK-ACK}
\end{frame}

\setbeamertemplate{background}{
  \parbox[c][.905\paperheight]{.93\paperwidth}{
    \vfill \hfill \pgfimage[height=.762\paperheight]{http-keepalive}
  }
}
\begin{frame}{\sn{2}Connection: keep-alive}
\end{frame}
\setbeamertemplate{background}{}

\begin{frame}[fragile]
\frametitle{\sn{2} Network latency \& TCP handshake in JavaScript}
\vspace{-.3cm}
\begin{semiverbatim}
\green<1>{var} t=[], tcp, rtt;
\green<1>{var} ld = \green<1>{function}() \{
   t.push(+\green<1>{new} \blue<1>{Date});
   \green<1>{if}(t.length > 2)  \brown<1>{// run 2 times}
     done();
   \green<1>{else} \{
     \green<1>{var} img = \green<1>{new} \blue<1>{Image};
     img.onload = ld;
     img.src="/1x1.gif?" + \blue<1>{Math}.random()
                         + '=' + \green<1>{new} \blue<1>{Date};
   \}
\};
\green<1>{var} done = \green<1>{function}() \{
  rtt=t[2]-t[1];
  tcp=t[1]-t[0]-rtt;
\};
ld();
\end{semiverbatim}
\end{frame}

\splashslide{Notice that we've ignored DNS lookup time here... how would you measure it?}

\leadinslide{3}{Network Throughput}

\splashslide[\sn{3}Measuring Network Throughput]{\LARGE{\( \frac{data\_length}{download\_time} \)}}

\begin{frame}[fragile]
\frametitle{\sn{3} Network Throughput in JavaScript}
\begin{semiverbatim}
\brown<1>{// Assume global object
// image=\{ url: ..., size: ... \}}
\green<1>{var} ts, rtt, bw, img = \green<1>{new} \blue<1>{Image};
img.onload=\green<1>{function}() \{
   rtt=(+\green<1>{new} \blue<1>{Date} - ts);
   bw = image.size*1000/rtt;    \brown<1>{// rtt is in ms}
\};
ts = +\green<1>{new} \blue<1>{Date};
img.src=image.url;
\end{semiverbatim}
\end{frame}

\splashslide[\sn{3}Measuring Network Throughput]{If it were that simple, I wouldn't be doing this talk.}

\splashimagev[0.85]{TCP1}{\sn{3}TCP Slow Start}

\splashslide[\sn{3}Measuring Network Throughput]{So to make the best use of bandwidth, we need resources that fit in a TCP window}

\splashimage[0.87]{bandwidth-latency}{\sn{3}There is no single size that will tax all available networks}{\tiny \href{http://www.yuiblog.com/blog/2010/04/08/analyzing-bandwidth-and-latency/}{http://www.yuiblog.com/blog/2010/04/08/analyzing-bandwidth-and-latency/}}

\begin{frame}[fragile]
\frametitle{\sn{3} Network Throughput in JavaScript -- Take 2}
\begin{semiverbatim}
\brown<1>{// image object is now an array of multiple images}
\green<1>{var} i=0;
\green<1>{var} ld = \green<1>{function}() \{
   \green<1>{if}(i>0)
      image[i-1].end = +\green<1>{new} \blue<1>{Date};
   \green<1>{if}(i >= image.length)
      done();
   \green<1>{else} \{
      \green<1>{var} img = \green<1>{new} \blue<1>{Image};
      img.onload = ld;
      image[i].start = +\green<1>{new} \blue<1>{Date};
      img.src=image[i].url;
   \}
   i++;
\};
\end{semiverbatim}
\end{frame}

\splashslide[\sn{3}Measuring Network Throughput]{Slow network connection, meet really huge image}

\begin{frame}[fragile]
\frametitle{\sn{3} Network Throughput in JavaScript -- Take 3}
\begin{semiverbatim}
   \green<1>{var} img = \green<1>{new} \blue<1>{Image};
   img.onload = ld;
   image[i].start = +\green<1>{new} \blue<1>{Date};
   image[i].timer =
         setTimeout(\green<1>{function}() \{
                       image[i].expired=true
                    \},
                    image[i].timeout);
   img.src=image[i].url;
\end{semiverbatim}
\end{frame}

\begin{frame}[fragile]
\frametitle{\sn{3} Network Throughput in JavaScript -- Take 3}
\begin{semiverbatim}
\green<1>{if}(i>0) \{
   image[i-1].end = +\green<1>{new} \blue<1>{Date};
   clearTimeout(image[i-1].timer);
\}
\end{semiverbatim}
\end{frame}

\begin{frame}[fragile]
\frametitle{\sn{3} Network Throughput in JavaScript -- Take 3}
\begin{semiverbatim}
\green<1>{if}(i >= image.length
      || (i > 0 \&\& image[i-1].expired)) \{
   done();
\}
\end{semiverbatim}
\end{frame}

\splashslide[\sn{3}Measuring Network Throughput]{Are we done yet?\\ \only<2->{sure...}}

\splashslide[\sn{3}Measuring Network Throughput]{Except network throughput is different every time you test it}

\splashslide[\sn{3}Measuring Network Throughput]{The code for that is NOT gonna fit on a slide}

\splashimage[0.85]{world-bandwidth}{\sn{3} Bandwidth is different around the world}

\leadinslide{4}{Page Load Time}

\begin{frame}
\frametitle{\sn{4} Page Load Time is Simple}
\begin{enumerate}
\item Get a timestamp when user initiates page request
\item Get a timestamp when page completes loading
\item Pass it through advanced Mathematics engine: $$t\_done = t\_end - t\_start$$
\end{enumerate}
\end{frame}

\splashslide[\sn{4}Page Load Time]{This is easy with Navigation Timing enabled browsers \\ \small (Except for all the versions with bugs)}

\splashslide[\sn{4}Page Load Time]{But what about legacy browsers?}

\splashslide[\sn{4}Page Load Time]{End time is measured in the \texttt{onload} or \texttt{pageshow} events}

\splashslide[\sn{4}Page Load Time]{Start time is measured in the \texttt{onbeforeunload} event}

\splashslide{But what if the user opens a link in a new tab?}

\splashslide[\sn{4}Page Load Time]{Start time is measured in the \texttt{onbeforeunload} or \texttt{onclick} or \texttt{onsubmit} events}

\splashslide{But not if the click happened on a FLASH object}

\splashslide[\sn{4}Page Load Time]{We also measure the \texttt{onunload} or \texttt{pagehide} events... \\ This gives us first byte time}

\splashslide[\sn{4}Page Load Time]{We cannot attach events for first page loads, but with enough data we can estimate using statistical analysis}

\splashslide{But... it's not over yet \\ \texttt{<link rel="prerender">}}

\splashslide[\sn{4}Page Load Time]{A pre-rendered page's \texttt{onload} event can fire before the user even clicks on the link}

\splashslide{boomerang takes care of this by measuring "click to visible", "request to load", "request to visible", and "load to visible"}

\leadinslide{5}{Resource Timing}

\splashslide{Using the Resource Timing API, get timing information for all resources on the page}

\splashslide{Yes, this results in very large beacons, so we use POST, and we will soon merge in a patch to compress the data}

\splashslide[And some hackery to capture XHR requests]{Create a wrapper around \texttt{window.XMLHttpRequest} and set timers on all its events}

\splashslide{Full code is at \href{https://github.com/lognormal/beyond-page-metrics/blob/gh-pages/src/instrument-xhr.js}{github.com/lognormal/beyond-page-metrics/}}

\leadinslide{6}{Other Stuff}

\begin{frame}{\sn{6} Other Stuff We Measure}
\begin{itemize}
  \item NavTiming, SPDY, FirstPaint -- \texttt{navtiming.js}
  \item \texttt{navigation.connection.*} -- \texttt{mobile.js}
  \item \texttt{window.performance.memory}  -- \texttt{memory.js} -- Chrome 22+ reporting now.
  \item Number of DOM nodes, images, scripts and byte size of HTML -- \texttt{memory.js}
\end{itemize}
\end{frame}

\splashslide{And we try to do it fast}

\leadinslide{7}{Keep it Clean}

\splashslide{We recently started capturing all JavaScript errors that happen inside boomerang, and beaconing them back}

\splashslide{So now we can find bugs as soon as it affects an end user}

\splashslide{We found that Firefox 31 started throwing exceptions on direct dereferences of \texttt{\green<1>{window}.\blue<1>{performance}} inside anonymous iframes \\ \tiny \href{https://bugzilla.mozilla.org/show_bug.cgi?id=1045096}{https://bugzilla.mozilla.org/show\_bug.cgi?id=1045096}}

\splashslide{We tried to use \\ \texttt{\blue<1>{if} (\green<1>{window}.hasOwnProperty("\gray<1>{performance}"))} but that doesn't work in IE10 because \texttt{\green<1>{window}} is a host object}

\splashslide{You have to use \\ \texttt{\blue<1->{if} ("\gray<1->{performance}" \blue<1->{in} \green<1->{window})} \\ which JSLint doesn't like...\\ \only<2->{But no one likes JSLint so it's okay}}

\splashslide{And speaking of IE, did you know that \\ \brown<1>{"TypeError: Permission Denied"} \\ is localized, so you have to capture every single language to detect it?}

\splashslide{On a more interesting note, IE11 running in compatibility mode can make it seem like IE7 has magically received Navigation Timing support}

\splashslide{We found that Chrome 38 started reporting 0 for \texttt{\blue<1->{firstPaintTime}}, and that helped the Chrome team realise, \\ "\gray<1->{This code (in Chrome)... we don't know how it ever worked}" \\ \tiny \href{https://code.google.com/p/chromium/issues/detail?id=422913}{https://code.google.com/p/chromium/issues/detail?id=422913}}

\splashslide{We also found a much older Chrome bug where cached content was reported as having a negative load time in Resource Timing}

\splashslide{We found that Mobile Safari's NavigationTiming implementation had weird bugs, but could not find a good pattern}

\splashslide{And Android 2.3's browser has a \texttt{\green<1>{document}.\blue<1>{createEvent}} method that exists, but always throws, so capability detection doesn't work nicely}

\splashimage[0.90]{ld50-geo}{As we found that users have different bugs too ;)}

\leadinslide{ni!}{din tur nu}

\leadinslide{--}{.done()}

\splashslide{Tack! \\ Fr\aa{}gor?}

\begin{frame}{Code/References}
  \begin{itemize}
    \item \texttt{http://lognormal.github.com/boomerang/doc/} (BSD Licensed)
    \item \href{http://www.lognormal.com}{www.lognormal.com}
  \end{itemize}
\end{frame}

\setbeamertemplate{background}{
  \parbox[c][\paperheight]{\paperwidth}{
    \vfill \hfill \pgfimage[height=\paperheight]{../bluesmoon}
  }
}
\begin{frame}
  \begin{itemize}
  \item Philip Tellis
  \item \href{http://twitter.com/bluesmoon}{@bluesmoon}
  \item \href{http://bluesmoon.info/}{philip@bluesmoon.info}
  \item \href{http://www.soasta.com/}{www.SOASTA.com}
  \item \href{http://lognormal.github.com/boomerang/doc/}{boomerang}
  \item \href{http://www.lognormal.com/blog/}{LogNormal Blog}
  \end{itemize}
\end{frame}
\setbeamertemplate{background}{}

\end{document}


