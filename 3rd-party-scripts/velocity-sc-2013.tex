\documentclass{beamer}

\mode<presentation>
{
  \usetheme{Copenhagen}

  \setbeamercovered{transparent}

}

\beamertemplatenavigationsymbolsempty

\definecolor{f2elblue}{RGB}{97,97,97}%{49,72,87}
\definecolor{dark-red}{RGB}{128,0,0}
\definecolor{dark-green}{RGB}{40,150,21}
\definecolor{light-gray}{gray}{0.85}
\definecolor{med-gray}{gray}{0.75}

\setbeamercolor{title in head/foot}{bg=orange!90!black,fg=white}
\setbeamercolor{author in head/foot}{bg=black,fg=white}
\setbeamercolor{subsection in head/foot}{bg=black!95,fg=white}
\setbeamercolor{structure}{fg=black}

\setbeamerfont{frametitle}{family=\rmfamily,shape=\itshape,size={\fontsize{14}{12}}}

\setbeamertemplate{blocks}[rounded][shadow=false]
\setbeamertemplate{items}[circle]

\setbeamertemplate{footline}%
{%
  \leavevmode%
  \hbox{%
    \begin{beamercolorbox}[wd=.45\paperwidth,ht=2.9ex,dp=1.125ex,leftskip=.3cm plus1fill,rightskip=.3cm]{author in head/foot}%
      \usebeamerfont{author in head/foot}\insertdate%
    \end{beamercolorbox}%
    \begin{beamercolorbox}[wd=.51\paperwidth,ht=2.9ex,dp=1.125ex,leftskip=.3cm,rightskip=.3cm plus1fil]{title in head/foot}%
      \usebeamerfont{title in head/foot}\insertshorttitle%
    \end{beamercolorbox}}%
    \begin{beamercolorbox}[wd=.04\paperwidth,ht=2.9ex,dp=1.125ex,center,leftskip=3pt plus1fill,rightskip=4pt]{author in head/foot}%
      \usebeamerfont{author in head/foot}\insertframenumber%
    \end{beamercolorbox}%
  \vskip0pt%
}%


\usepackage[english]{babel}

\usepackage[latin1]{inputenc}

\usepackage{times}
\usepackage[T1]{fontenc}

\usepackage{ulem}

\usepackage{mathtools}

\author{Philip Tellis / \texttt{philip@bluesmoon.info}}

\title{Improving 3rd Party Script Performance with <IFRAME>s}

\date{\href{http://velocityconf.com/velocity2013/public/schedule/detail/28313}{Velocity 2013} / 2013-06-20}


\pgfdeclareimage[height=0.5cm]{bluesmoon-logo}{../ui/default/bodybg}
\pgfdeclareimage[height=11pt]{cc-licence}{../by-nc-sa-3.0-88x31}
\logo{\pgfuseimage{cc-licence}}


%%%% Begin defining new commands for this template

% Serial number in slide title
\newcommand{\sn}[1]{\textrm{\textit{\Huge{\raisebox{-3pt}[4pt][8pt]{\textcolor{f2elblue}{#1}}}}}\hspace{4pt}}
% large Centered Italic Roman text within a slide (building block for the following)
\newcommand{\innersplash}[1]{
  \begin{center}
    \Large \textrm{\textit{ #1 } }
  \end{center}
}
% Slide that has just one large centered thing to say
\newcommand{\splashslide}[2][{}]{
  \begin{frame}
  \frametitle{#1}
  \innersplash{#2}
  \end{frame}
}
% Leadin slide at the start of an exploit
\newcommand{\leadinslide}[2]{
  \splashslide{
     {\fontsize{150}{20}\selectfont{\raisebox{0pt}[90pt][0pt]{\textcolor{light-gray}{#1}}}} \\ \huge \textcolor{gray}{#2}
  }
}
% Suggested fix at the end of an exploit
\newcommand{\suggestedfix}[2]{
  \splashslide[\sn{#1} Suggested fix]{{#2}}
}
% Syntax highlighting
\def\gray<#1>#2{\textcolor<#1>{gray}{\textbf<#1>{#2}}}
\def\brown<#1>#2{\textcolor<#1>{brown}{\textbf<#1>{#2}}}
\def\green<#1>#2{\textcolor<#1>{dark-green}{\textbf<#1>{#2}}}
\def\blue<#1>#2{\textcolor<#1>{blue}{\textbf<#1>{#2}}}
\def\purple<#1>#2{\textcolor<#1>{purple}{\textbf<#1>{#2}}}
\def\red<#1>#2{\textcolor<#1>{red}{\textbf<#1>{#2}}}

\newcommand{\textsubscript}[1]{\ensuremath{_{\textrm{#1}}}}
%%%% End of new commands




\begin{document}

\begin{frame}
  \titlepage
\end{frame}

\setbeamertemplate{background}{
  \parbox[c][\paperheight]{\paperwidth}{
    \vfill \hfill \pgfimage[height=\paperheight]{../bluesmoon}
  }
}
\begin{frame}
  \begin{itemize}
  \item Philip Tellis
  \item \href{http://twitter.com/bluesmoon}{@bluesmoon}
  \item \href{http://bluesmoon.info/}{philip@bluesmoon.info}
  \item \href{http://www.soasta.com/}{SOASTA}
  \end{itemize}
\end{frame}
\setbeamertemplate{background}{}

\splashslide{Do you use JavaScript?}

\splashslide{\texttt{<\green<1>{script} src="..."></\green<1>{script}>}}
% The traditional way to load scripts onto a page
% Do this if you control everything about this script and your page completely depends on it for functionality

\begin{frame}{<script src>}
\begin{itemize}
  \item Works well with browser lookahead
  \item But blocks everything
  \item Yes, you can use async or defer
\end{itemize}
\end{frame}

\setbeamertemplate{background}{
  \parbox[c][\paperheight]{\paperwidth}{
    \vfill \hfill \pgfimage[height=\paperheight]{3rd-party-monsters}
  }
}
\begin{frame}
\end{frame}

\setbeamertemplate{background}{
  \parbox[c][\paperheight]{\paperwidth}{
    \vfill \hfill \pgfimage[height=\paperheight]{javascript-files-block}
  }
}
\begin{frame}
\end{frame}
\setbeamertemplate{background}{}

\splashslide{\texttt{document.createElement("script");}}

\begin{frame}{dynamic script node}
\begin{enumerate}
  \item Loads in parallel with the rest of the page
  \item Still blocks the \texttt{onload} event
  \item No telling when it will load up
\end{enumerate}
\end{frame}

\splashslide{The Method Queue Pattern}

\begin{frame}[fragile]{MQP}
\begin{semiverbatim}
\green<1>{var} \_mq = \_mq || [];

\green<1>{var} s = \blue<1>{document}.createElement(\brown<1>{"script"}),
    t = \blue<1>{document}.getElementsByTagName(\brown<1>{"script"})[0];

s.src=\brown<1>{"http://some.site.com/script.js"};
t.parentNode.insertBefore(s, t);

\gray<1>{// script.js will be available some time in the}
\gray<1>{// future, but we can call its methods}

\_mq.push([\brown<1>{"method1"}, list, of, params]);
\_mq.push([\brown<1>{"method2"}, other, params]);
\end{semiverbatim}
\end{frame}

\begin{frame}[fragile]{MQP}
\begin{semiverbatim}
\green<1>{var} self = \blue<1>{this};
\_mq = \_mq || [];
\green<1>{while}(\_mq.length) \{
    \gray<1>{// remove the first item from the queue}
    \green<1>{var} params = \_mq.shift();
    \gray<1>{// remove the method from the first item}
    \green<1>{var} method = params.shift();

    self[method].apply(self, params);
\}

\_mq.push = function(params) \{
    \gray<1>{// remove the method from the first item}
    \green<1>{var} method = params.shift();

    self[method].apply(self, params);
\}
\end{semiverbatim}
\end{frame}

\splashslide{That takes care of \#3}

% http://www.flickr.com/photos/memestate/54408373/
\setbeamertemplate{background}{
  \parbox[c][\paperheight]{\paperwidth}{
    \vfill \hfill \pgfimage[height=\paperheight]{stop-hammertime}
  }
}
\begin{frame}{But we still block \texttt{onload}}
\end{frame}
\setbeamertemplate{background}{}

\splashslide{IFRAMEs to the rescue}

\splashslide{But IFRAMEs block onload until the subpage has loaded}

\splashslide{(This sub-page intentionally left blank)}

\begin{frame}[fragile]{So here's the code -- Section I}
\begin{semiverbatim}
\gray<1>{// Section 1 -- Create the iframe}
\green<1>{var} dom,doc,where,
    iframe = \blue<1>{document}.createElement(\brown<1>{"iframe"});

iframe.src = \brown<1>{"javascript:false"};
iframe.title = \brown<1>{""}; iframe.role = \brown<1>{"presentation"};
(iframe.frameElement || iframe).style.cssText =
    \brown<1>{"width: 0; height: 0; border: 0"};

where = \blue<1>{document}.getElementsByTagName(\brown<1>{"script"});
where = where[where.length - 1];
where.parentNode.insertBefore(iframe, where);
\end{semiverbatim}
\begin{center} \textrm{\textit{\small (Note that we set \texttt{iframe.title} and \texttt{iframe.role} to avoid hurting screenreaders)}} \end{center}
\end{frame}


\splashslide{\texttt{javascript:false} is key to solving most cross-domain issues \\ \vspace{2 cm} \tiny \texttt{about:blank} is problematic on IE6 with SSL}

\splashslide{Except if the page developer sets \texttt{document.domain}}

\setbeamertemplate{background}{
  \parbox[c][\paperheight]{\paperwidth}{
    \pgfimage[height=\paperheight]{allspaw-wtf}
  }
}
\begin{frame}
\end{frame}
\setbeamertemplate{background}{}


\begin{frame}[fragile]{The code -- Section II}
\begin{semiverbatim}
\gray<1>{// Section 2 -- handle document.domain}
\green<1>{try} \{
    \gray<1>{// sec exception if document.domain was set}
    doc = iframe.contentWindow.document;
\}
\green<1>{catch}(e) \{
    dom = \blue<1>{document}.domain;
    iframe.src = 
        \brown<1>{"javascript:\green<1>{var} d=\blue<1>{document}.open();"} +
        \brown<1>{"d.domain='"} + dom + \brown<1>{"';"} + 
        \brown<1>{"\blue<1>{void}(0);"};
    doc = iframe.contentWindow.document;
\}
\end{semiverbatim}
\end{frame}

\splashslide{Only set \texttt{document.domain} if it has already been set!}

\begin{frame}[fragile]{The code -- Section III}
\begin{semiverbatim}
\gray<1>{// Section 3 -- tell the iframe to load our script}

doc.open().\_l = function() \{
    \green<1>{var} js = \blue<1>{this}.createElement(\brown<1>{"script"});
    \green<1>{if}(dom)
        \blue<1>{this}.domain = dom;
    js.id = \brown<1>{"js-iframe-async"};
    js.src = \textbf{script\_url};
    \blue<1>{this}.body.appendChild(js);
\};

doc.write(\brown<1>{'<body onload="document.\_l();">'});
doc.close();
\end{semiverbatim}
\end{frame}

\splashslide{Notice that we've set \texttt{document.domain} again}

\splashslide{Inside this function, \blue<1>{\texttt{document}} is the parent document and \blue<1>{\texttt{this}} is the iframe!}

\splashslide{Also, global variables inside \texttt{\_l()} are global to the parent window}

\begin{frame}[fragile]{Modify the MQP for IFRAME support}
\begin{semiverbatim}
GLOBAL = \blue<1>{window};

\gray<1>{// Running in an iframe, and our script node's
// id is js-iframe-async}
\green<1>{if}(\blue<1>{window}.parent != \blue<1>{window} &&
   \blue<1>{document}.getElementById(\brown<1>{"js-iframe-async"})) \{

    GLOBAL = \blue<1>{window}.parent;
\}

GLOBAL.\_mq = GLOBAL.\_mq || [];
\_mq = GLOBAL.\_mq;
\end{semiverbatim}
\end{frame}

\splashslide{\texttt{GLOBAL} refers to the parent window and \texttt{window} refers to the iframe}

\splashslide{So attach events to \texttt{GLOBAL}}

\begin{frame}{Summary}
\begin{itemize}
  \item Create an iframe with src set to \texttt{javascript:false}
  \item Set \texttt{\blue<1>{document}.domain} if needed (twice)
  \item Write dynamic script node into iframe on iframe's \texttt{onload} event
  \item Alias parent window into iframe
\end{itemize}
\end{frame}

\setbeamertemplate{background}{
  \parbox[c][\paperheight]{\paperwidth}{
    \pgfimage[height=\paperheight]{klein-happy}
  }
}
\begin{frame}{Result: Happy Customers}
\end{frame}
\setbeamertemplate{background}{}

\begin{frame}{Read all about it}
\begin{itemize}
  \item \href{http://www.lognormal.com/blog/2012/12/12/the-script-loader-pattern/}{http://lognormal.com/blog/2012/12/12/the-script-loader-pattern/}
  \item \href{https://www.w3.org/Bugs/Public/show_bug.cgi?id=21074}{https://www.w3.org/Bugs/Public/show\_bug.cgi?id=21074}
\end{itemize}
\end{frame}


\splashslide{Cache busting with a far-future expires header}

\splashslide{Some more code...}

\splashslide{\texttt{\blue<1>{location}.reload(\brown<1>{true});}}

\begin{frame}[fragile]{More completely}
\begin{semiverbatim}
<script src="SCRIPT.js"></script>
<script>
\green<1>{var} reqd\_ver = \blue<1>{location}.search;

\blue<1>{window}.onload = \green<1>{function}() \{
    \green<1>{var} ver = SCRIPT.version;
    \green<1>{if} (ver < reqd\_ver) \{
        \blue<1>{location}.reload(\brown<1>{true});
    \}
\};
</script>
\end{semiverbatim}

The condition protects us from an infinite loop with bad proxies and Firefox 3.5.11

\small Note: Don't use \texttt{\blue<1>{location}.hash} -- it messes with history on IE8.
\end{frame}

\splashslide[And the blog post...]{\href{http://www.lognormal.com/blog/2012/06/17/more-on-updating-boomerang/}{http://www.lognormal.com/blog/2012/06/17/more-on-updating-boomerang/}}

% http://www.flickr.com/photos/rj3/5635837522/
\setbeamertemplate{background}{
  \parbox[c][\paperheight]{\paperwidth}{
    \pgfimage[height=\paperheight]{zakas}
  }
}
\begin{frame}{Join us right after this guy...}
\end{frame}
\setbeamertemplate{background}{}

\begin{frame}{Revolutionary RUM}
  \begin{enumerate}
    \item Combine 1oz Cane Rum, 3/4oz Blood Orange Puree, \& 1/2oz citrus infused syrup in a mixing glass
    \item Add ice and shake vigourously
    \item Strain into a chilled cocktail glass
    \item Top with lime-mint foam
  \end{enumerate}

\vspace{1 cm}
17:45, Evolution Caf\'{e} + Bar, Hyatt Lobby
\end{frame}

\setbeamertemplate{background}{
  \parbox[c][\paperheight]{\paperwidth}{
    \vfill \hfill \pgfimage[height=\paperheight]{../bluesmoon}
  }
}
\begin{frame}
  \begin{itemize}
  \item Philip Tellis
  \item \href{http://twitter.com/bluesmoon}{@bluesmoon}
  \item \href{http://bluesmoon.info/}{philip@bluesmoon.info}
  \item \href{http://www.soasta.com/}{www.SOASTA.com}
  \item \href{http://lognormal.github.com/boomerang/doc/}{boomerang}
  \item \href{http://www.lognormal.com/blog/}{LogNormal Blog}
  \end{itemize}
\end{frame}
\setbeamertemplate{background}{}

\begin{frame}{Image Credits}
\begin{itemize}
  \item Stop Hammertime by Rich Anderson \\ \small \href{http://www.flickr.com/photos/memestate/54408373/}{http://www.flickr.com/photos/memestate/54408373/}
  \item Nicholas Zakas by Ben Alman \\ \small \href{http://www.flickr.com/photos/rj3/5635837522/}{http://www.flickr.com/photos/rj3/5635837522/}
  \item All other images taken at Velocity 2013 by Philip Tellis and shared under a Creative Commons Attribution-Share Alike License.
\end{itemize}
\end{frame}

\end{document}


